%%%%%%%%%%%%%%%%%%%%%%%%%%%%%%%%%%%%%%%%%
% Beamer Presentation
% LaTeX Template
% Version 1.0 (10/11/12)
%
% This template has been downloaded from:
% http://www.LaTeXTemplates.com
%
% License:
% CC BY-NC-SA 3.0 (http://creativecommons.org/licenses/by-nc-sa/3.0/)
%
%%%%%%%%%%%%%%%%%%%%%%%%%%%%%%%%%%%%%%%%%

%----------------------------------------------------------------------------------------
%	PACKAGES AND THEMES
%----------------------------------------------------------------------------------------

\documentclass{beamer}

\mode<presentation> {

% The Beamer class comes with a number of default slide themes
% which change the colors and layouts of slides. Below this is a list
% of all the themes, uncomment each in turn to see what they look like.

%\usetheme{default}
%\usetheme{AnnArbor}
%\usetheme{Antibes}
%\usetheme{Bergen}
%\usetheme{Berkeley}
%\usetheme{Berlin}
%\usetheme{Boadilla}
%\usetheme{CambridgeUS}
%\usetheme{Copenhagen}
%\usetheme{Darmstadt}
%\usetheme{Dresden}
%\usetheme{Frankfurt}
%\usetheme{Goettingen}
%\usetheme{Hannover}
%\usetheme{Ilmenau}
%\usetheme{JuanLesPins}
%\usetheme{Luebeck}
\usetheme{Madrid}
%\usetheme{Malmoe}
%\usetheme{Marburg}
%\usetheme{Montpellier}
%\usetheme{PaloAlto}
%\usetheme{Pittsburgh}
%\usetheme{Rochester}
%\usetheme{Singapore}
%\usetheme{Szeged}
%\usetheme{Warsaw}

% As well as themes, the Beamer class has a number of color themes
% for any slide theme. Uncomment each of these in turn to see how it
% changes the colors of your current slide theme.

%\usecolortheme{albatross}
\usecolortheme{beaver} %nice
%\usecolortheme{beetle}
%\usecolortheme{crane} %nice
%\usecolortheme{dolphin} %nice
%\usecolortheme{dove}
%\usecolortheme{fly}
%\usecolortheme{lily} %nice
%\usecolortheme{orchid} %nice
%\usecolortheme{rose}
%\usecolortheme{seagull}
%\usecolortheme{seahorse}
%\usecolortheme{whale}
%\usecolortheme{wolverine}

%\setbeamertemplate{footline} % To remove the footer line in all slides uncomment this line
%\setbeamertemplate{footline}[page number] % To replace the footer line in all slides with a simple slide count uncomment this line

%\setbeamertemplate{navigation symbols}{} % To remove the navigation symbols from the bottom of all slides uncomment this line
}

\usepackage{graphicx} % Allows including images
\usepackage{booktabs} % Allows the use of \toprule, \midrule and \bottomrule in tables
\usepackage[belowskip=0pt,aboveskip=0pt]{caption}
\usepackage{amsmath}
\usepackage{pifont}
\usepackage{amsthm}
\usepackage{caption}
\usepackage[]{epstopdf}
\usepackage{braket}
\usepackage{color}
\usepackage{mathtools}


\usepackage{listings}
\lstdefinestyle{mystyle}{
%    backgroundcolor=\color{backcolour},   
%    commentstyle=\color{codegreen},
%    keywordstyle=\color{magenta},
%    numberstyle=\tiny\color{codegray},
%    stringstyle=\color{codepurple},
    basicstyle=\footnotesize,
    breakatwhitespace=false,         
    breaklines=true,                 
    captionpos=b,                    
    keepspaces=true,                 
    numbers=left,                    
    numbersep=5pt,                  
    showspaces=false,                
    showstringspaces=false,
    showtabs=false,                  
    tabsize=2
}
 
\lstset{style=mystyle}


\usepackage{caption}

\captionsetup[sub]{font=scriptsize,labelfont={}}

\providecommand{\abs}[1]{\lvert#1\rvert}


\usepackage{xcolor}
\definecolor{dgreen}{rgb}{0.,0.6,0.}
\definecolor{dred}{rgb}{1.,0.5,0.5}

\newcommand*{\Hhat}{\skew{5}{\hat}{H}}



%\setlength{\intextsep}{5pt plus 2pt minus 2pt}
%\setlength{\textfloatsep}{5pt plus 2pt minus 2pt}
%\setlength{\floatsep}{5pt plus 2pt minus 2pt}
\setbeamertemplate{frametitle}[default][center]
%----------------------------------------------------------------------------------------
%	TITLE PAGE
%----------------------------------------------------------------------------------------
\title[APPM 7400: HW\#1]{Hexagonal Grids for solving PDEs} % The short title appears at the bottom of every slide, the full title is only on the title page

\author[Prasanth Prahladan]{Prasanth Prahladan} % Your name
\institute[CU Boulder] % Your institution as it will appear on the bottom of every slide, may be shorthand to save space
{University of Colorado Boulder  \\ % Your institution for the title page
\medskip
%\textit{prasanth.prahladan@gmail.com} % Your email address
}
\date{\today} % Date, can be changed to a custom date

%\centering
%\titlegraphic{
%   \includegraphics[width=2cm]{Figures/iitm_logo.eps}
%}

\begin{document}
\scriptsize

\begin{frame}
\titlepage % Print the title page as the first slide
\end{frame}

%----------------------------------------------------------------------------------------
%	PRESENTATION SLIDES: 1
%----------------------------------------------------------------------------------------

\begin{frame}
\frametitle{Objectives}
\begin{enumerate}
\item Construction of Spatial Grid
\item Grids and Grid Function
\item Tilings in Wave-number Space
\item Finite Difference Operators
\item Explicit Finite Difference Schemes
\item Stability Conditions 
\item Numerical Dispersion
\item Example: Use of Staggered Grids in Computational Optics/Electro-magnetics
\end{enumerate}

\end{frame}


%----------------------------------------------------------------------------------------
%	PRESENTATION SLIDES
%----------------------------------------------------------------------------------------
\begin{frame}
\frametitle{Introduction to Spatial Grid}
\begin{figure}
%\vspace*{-1.5cm}
%\centering
\includegraphics[scale=0.2]{./images/jpgSpatialGrid.jpg}
\label{fig:spatialGrid}
\end{figure}

A regular 2-D spatial grid is a collection of points defined as:
\begin{align*}
\mathbf{G}_h = \bigg\{\mathbf{r}_{m_1, m_2} = h \big( m_1 \mathbf{x_1} + m_2 \mathbf{x_2} \big) | (m_1,m_2) \in \mathbb{Z}^2\bigg\}
\end{align*}

A hexagonal grid is obtained when we displace each layer of points by $(h_x, h_y) = (\frac{h}{2}, \frac{h\sqrt{3}}{2})$.
\end{frame}

%----------------------------------------------------------------------------------------

\begin{frame}
\frametitle{Spatial Grids}

For each regular lattice, a suitable coordinate axis may be chosen for facilitating analysis. 
For the Rectilinear Grid, we have $\big( \mathbf{x_1}, \mathbf{x_2} \big) = \big([1,0]^T, [0,1]^T \big)$.
\begin{figure}
\centering
\begin{minipage}{.5\textwidth}
  \centering
\includegraphics[scale=0.2]{./images/jpgRect.jpg}
\label{fig:RectilinearGrid}
\captionof{figure}{Rectilinear Grid}
\end{minipage}%
\begin{minipage}{.5\textwidth}
  \centering
\includegraphics[scale=0.2]{./images/jpgHex.jpg}
\label{fig:HexGrid}\
\captionof{figure}{Hexagonal Grid}
\end{minipage}


\end{figure}

For the Hexagonal Grid, we have $\big( \mathbf{x_1}, \mathbf{x_2} \big) = \big([1,0]^T, [\frac{-1}{2},\frac{\sqrt{3}}{2}]^T \big)$.

\end{frame}

%----------------------------------------------------------------------------------------

\begin{frame}
\frametitle{Staggered Grids}

\begin{figure}
\centering
\begin{minipage}{.5\textwidth}
  \centering
\includegraphics[scale=0.2]{./images/jpgStaggeredColocated.jpg}
\label{fig:StaggeredColocated}
  \captionof{figure}{Staggered Colocated Hexagonal Grids}
\end{minipage}%
\begin{minipage}{.5\textwidth}
  \centering
\includegraphics[scale=0.2]{./images/jpgHexGridStruc.jpg}
\label{fig:HexGridStructure}
\captionof{figure}{Hexagonal Grid Structure}
\end{minipage}
\end{figure}



\end{frame}

\begin{frame}
\frametitle{Triangular and Hexagonal Nets}
\tiny
\begin{minipage}{.5\textwidth}
  \centering
  Triangular Net
\begin{align*}
u_1 - u_0 &= u(x+h,y) - u(x,y) \\
&= h (\frac{\partial}{\partial x})u + \frac{h^2}{2!}(\frac{\partial^2 }{\partial^2 x})u + \frac{h^3}{3!} (\frac{\partial^3}{\partial^3 x})u + \cdots\\
u_2 - u_0 &= u(x+\frac{h}{2},y+\frac{\sqrt{3}h}{2}) - u(x,y) \\
&= h \bigg(\frac{1}{2}\frac{\partial}{\partial x} + \frac{\sqrt{3}}{2} \frac{\partial}{\partial y} \bigg)u \\
&+ \frac{h^2}{2!}\bigg(\frac{1}{2}\frac{\partial}{\partial x} + \frac{\sqrt{3}}{2} \frac{\partial}{\partial y} \bigg)^2 u \\
&+ \frac{h^3}{3!} \bigg(\frac{1}{2}\frac{\partial}{\partial x} + \frac{\sqrt{3}}{2} \frac{\partial}{\partial y} \bigg)^3 u + \cdots\\
u_3 - u_0 &= u(x-\frac{h}{2},y+\frac{\sqrt{3}h}{2}) - u(x,y) \\
&= h \bigg(\frac{-1}{2}\frac{\partial}{\partial x} + \frac{\sqrt{3}}{2} \frac{\partial}{\partial y} \bigg)u \\
&+ \frac{h^2}{2!}\bigg(\frac{-1}{2}\frac{\partial}{\partial x} + \frac{\sqrt{3}}{2} \frac{\partial}{\partial y} \bigg)^2 u \\
&+ \frac{h^3}{3!} \bigg(\frac{-1}{2}\frac{\partial}{\partial x} + \frac{\sqrt{3}}{2} \frac{\partial}{\partial y} \bigg)^3 u + \cdots\\
\cdots (u_4-u_0), &(u_5-u_0), (u_6-u_0)
\end{align*}
\end{minipage}%
\begin{minipage}{.5\textwidth}
  \centering
  Hexagonal Net
\begin{align*}
u_1 - u_0 &= u(x+\frac{h}{2},y+\frac{\sqrt{3}h}{2}) - u(x,y) \\
&= h \bigg(\frac{1}{2}\frac{\partial}{\partial x} + \frac{\sqrt{3}}{2} \frac{\partial}{\partial y} \bigg)u \\
&+ \frac{h^2}{2!}\bigg(\frac{1}{2}\frac{\partial}{\partial x} + \frac{\sqrt{3}}{2} \frac{\partial}{\partial y} \bigg)^2 u \\
&+ \frac{h^3}{3!} \bigg(\frac{1}{2}\frac{\partial}{\partial x} + \frac{\sqrt{3}}{2} \frac{\partial}{\partial y} \bigg)^3 u + \cdots\\
u_2 - u_0 &= u(x-h,y) - u(x,y) \\
&= -h (\frac{\partial}{\partial x})u + \frac{h^2}{2!}(\frac{\partial^2 }{\partial^2 x})u + \frac{-h^3}{3!} (\frac{\partial^3}{\partial^3 x})u + \cdots\\
u_3 - u_0 &= u(x+\frac{h}{2},y-\frac{\sqrt{3}h}{2}) - u(x,y) \\
&= h \bigg(\frac{1}{2}\frac{\partial}{\partial x} + \frac{-\sqrt{3}}{2} \frac{\partial}{\partial y} \bigg)u \\
&+ \frac{h^2}{2!}\bigg(\frac{1}{2}\frac{\partial}{\partial x} + \frac{-\sqrt{3}}{2} \frac{\partial}{\partial y} \bigg)^2 u \\
&+ \frac{h^3}{3!} \bigg(\frac{1}{2}\frac{\partial}{\partial x} + \frac{-\sqrt{3}}{2} \frac{\partial}{\partial y} \bigg)^3 u + \cdots\\
\end{align*}
\end{minipage}
\end{frame}


\begin{frame}
\frametitle{Triangular and Hexagonal Nets}
\tiny

\centering
\begin{minipage}{.5\textwidth}
  \centering
Triangular Net
% \sum_{i=1}^{6} u_i - 6 u_0 &= \frac{3h^2}{2!} \Delta u + \frac{(3/2 h^2)^2}{4} + \cdots \notag \\
\begin{align}
\frac{2}{3h^2}(\sum u_i - 6 u_0) &= \Delta u + \frac{h^2}{16} \Delta^2 u \notag \\
&+ R_0 \bigg(O(h^4)O\big( \frac{\partial^6}{\partial^k x \partial^{6-k}y}\big)\bigg)
\end{align}
\end{minipage}%
\begin{minipage}{.5\textwidth}
  \centering
Hexagonal Net
\begin{align}
\frac{4}{3h^2} \bigg( \sum_{i=1}^3 u_i - 3 u_0 \bigg) &= \Delta u \notag \\
&+ R_0 \bigg( O(h) O(\frac{\partial^3}{\partial^k x \partial^{3-k} y})\bigg)
\end{align}
\end{minipage}

\end{frame}



%----------------------------------------------------------------------------------------

\begin{frame}
\frametitle{Grid-functions for solving PDEs}

\begin{figure}
%\vspace*{-1.5cm}
%\centering
\includegraphics[scale=0.2]{./images/jpgGridFunction.jpg}
\label{fig:GridFunctions}
\end{figure}

We define a grid function $u^{n}_{m_1, m_2}$ as a time series at each point on the spatial grid which approximates the continuous functions $u(t,x,y)$ at time $t = nk$, where $k$ is the time-step and at the spatial position $(x,y)=\mathbf{r}_{m_1,m_2}$.

\end{frame}
%----------------------------------------------------------------------------------------

\begin{frame}
\frametitle{Finite Difference Operators}

In the FDTD method, differential operators are approximated by finite difference operators. 

First, we define the Unit-Shift operators as follows:
\begin{align}
S_{t\pm}\big( u^n_{m_1,m_2}\big) &= \big( u^{n+1}_{m_1,m_2}\big)\\
S_{x_1 \pm}\big( u^{n}_{m_1,m_2}\big) &= \big( u^{n}_{m_1 \pm 1,m_2}\big)\\
S_{x_2 \pm}\big( u^{n}_{m_1,m_2}\big) &= \big( u^{n}_{m_1,m_2 \pm 1}\big)\\
S_{x_3 \pm} &= S_{x_2 \mp} \cdot S_{x_2 \mp}
\end{align}

Next, we proceed to build second-order finite difference operators as:
\begin{align}
\delta_{tt} &= \frac{1}{k^2} \big( S_{t-} + S_{t+} -2 \big)\\
\delta_{x_i x_i} &= \frac{1}{h^2} \big( S_{x_i -} + S_{x_i +} -2 \big)\\
\end{align}
\end{frame}

\begin{frame}
\frametitle{Finite Difference Operators}
On the hexagonal grid we employ seven-points to build a second-order accurate approximation to the Laplacian:
\begin{align}
\delta_{\Delta \text{HEX}} &= \frac{2}{3} \big( \delta_{x_1 x_1} + \delta_{x_2 x_2} + \delta_{x_3 x_3}\big) \notag \\
&= \Delta + \frac{h^2}{16} \Delta^2 + O(h^4)
\end{align}
where $\Delta = \frac{2}{3}\big( \frac{\partial^2}{\partial x_1^2} + \frac{\partial^2}{\partial x_2^2} + \frac{\partial^2}{\partial x_3^2}\big)$ is the 2-D Laplacian operator for the 2-D Wave Equation
\begin{align}
\bigg(\frac{\partial^2}{\partial t^2} - c^2 \Delta \bigg) u = 0
\end{align}
\end{frame}
%----------------------------------------------------------------------------------------
\begin{frame}
\frametitle{Finite Difference Operators}
Using the finite-difference operators, we solve the approximate finite-difference 2-D Wave Equation
\begin{align}
\bigg(\frac{\partial^2}{\partial t^2} - c^2 \Delta \bigg) u^{n}_{m_1,m_2} = 0
\end{align}

using the explicit update equation
\begin{align}
u^{n+1}_{m_1,m_2} &= \frac{2\mu^2 }{3} \big( u^{n}_{m_1+1,m_2} + u^{n}_{m_1-1,m_2} + u^{n}_{m_1,m_2+1} \\
&+ u^{n}_{m_1,m_2-1} + u^{n}_{m_1+1,m_2+1} + u^{n}_{m_1-1,m_2-1} \big) \\
&+ (2-4\mu^2) u^{n}_{m_1,m_2} - u^{n-1}_{m_1,m_2}
\end{align}
where, 
\begin{align}
\mu = ck/h \label{eq:CourantNumber}
\end{align}
$\mu$ is the Courant Number, which is the ratio between the time step and the grid spacing for a given wave speed. 
\end{frame}



\begin{frame}
\frametitle{Stability Conditions}
\begin{figure}
\centering
\begin{minipage}{.5\textwidth}
  \centering
\includegraphics[scale=0.1]{./images/jpgDualRect.jpg}
\label{fig:DualRectGrid}
  \captionof{figure}{Dual-Rectilinear Grid}
\end{minipage}%
\begin{minipage}{.5\textwidth}
  \centering
\includegraphics[scale=0.1]{./images/jpgDualHex.jpg}
\label{fig:DualHexGrid}
\captionof{figure}{Dual-Hexagonal Grid}
\end{minipage}
\end{figure}

Determine the maximum value of Courant Number, $\mu = ck/h$ such that no exponentially growing plane-wave solutions of the form
\begin{align*}
u^{n}_{m_1, m_2} = e^{jk n\omega } e^{jh (\xi_x, \xi_y)\cdot(m_1 \mathbf{x_1}+ m_2 \mathbf{x_2})}
\end{align*}
where $(\omega, \xi_x, \xi_y) \in \mathbb{C}^3$ are the wave-numbers in complex frequency domain, $(h,k) \in  \mathbb{R}^2$ grid spacing and the integral grid locations $(n,m_1,m_2) \in \mathbb{Z}^3$.

\end{frame}

\begin{frame}
\frametitle{Stability Conditions}

Stability Conditions:
\begin{enumerate}
\item $\omega \in [0,\pi]$
\item $(\xi_x, \xi_y) \in \{$ One Wave-Number Cell $\}$ of the Grid. 
\end{enumerate}


For the Hexagonal Grid, it is determined to be 
\begin{align}
\mu \leq \sqrt{\frac{2}{3}}
\end{align}
which, is achieved at the corners of the hexagon. By searching over $ \xi_x, \xi_y \in [-\frac{4\pi}{3h},\frac{4\pi}{3h}]$ is sufficient to cover the entire the hexagonal wave-number cell. 
However, considering hexagonal tiling of the wave-numbers is not necessary in determining stability condition. It is necessary for determining Numerical Dispersion of the scheme.
\end{frame}

%----------------------------------------------------------------------------------------

%----------------------------------------------------------------------------------------

\begin{frame}
\frametitle{Application of Staggered Spatial Grids in Optics/Electro-magnetics}

\tiny

\begin{figure}
\centering
\begin{minipage}{.5\textwidth}
  \centering
\includegraphics[scale=0.2]{./images/jpgUnStaggered.jpg}
\label{fig:StaggeredColocated}
\end{minipage}%
\begin{minipage}{.5\textwidth}
  \centering
\includegraphics[scale=0.2]{./images/jpgHexGridStruc.jpg}
\end{minipage}
\end{figure}

\begin{minipage}{.5\textwidth}
  \centering
\begin{align*}
\frac{d D_{z0}}{dt} &= \frac{1}{6h}\big( 2 H_{y6} - 2 H_{y3} + H_{y1} - H_{y4} + H_{x5} - H_{x2} \big) \\
&- \sqrt{3}\big(H_{x1}- H_{x5} + H_{x2} - H_{x4}\big)\\
\end{align*}
\end{minipage}%
\begin{minipage}{.5\textwidth}
  \centering
\begin{align*}
\frac{d B_{x0}}{dt} &= \frac{-\sqrt{3}}{6h} \big( E_{z1} - E_{z4} + E_{z2} - E_{z5}  \big)\\
\frac{d B_{y0}}{dt}&= \frac{1}{6h} \big( 2E_{z6} -2E_{z3} E_{z1} - E_{z4} + E_{z2} - E_{z5}  \big)
\end{align*}
\end{minipage}
\end{frame}


\begin{frame}
\frametitle{Application of Staggered Spatial Grids in Optics/Electro-magnetics}
\tiny
\begin{figure}
\centering
\begin{minipage}{.5\textwidth}
  \centering
\includegraphics[scale=0.2]{./images/jpgStaggeredUncolocated.jpg}
\label{fig:StaggeredColocated}
\end{minipage}%
\begin{minipage}{.5\textwidth}
  \centering
\includegraphics[scale=0.2]{./images/jpgHexGridStruc.jpg}
\end{minipage}
\end{figure}

\begin{minipage}{.5\textwidth}
  \centering
\begin{align*}
\frac{d D_{z0}}{dt} &= \frac{2}{3h} \big( H^a_1 - H^d_1 + H^b_2 - H^b_e + H^c_3 - H^f_3 \big)\\
\frac{d B^a_1}{dt} &= \frac{1}{k} \big(E_{z5} - E_{z0} \big)\\
\end{align*}
\end{minipage}%
\begin{minipage}{.5\textwidth}
  \centering
\begin{align*}
\frac{d B^b_2}{dt}&= \frac{1}{k}\big( E_{z6} - E_{z0}\big)\\
\frac{d B^c_3}{dt}&= \frac{1}{k}\big( E_{z1} - E_{z0} \big)
\end{align*}
\end{minipage}

\end{frame}

%----------------------------------------------------------------------------------------
\begin{frame}
\frametitle{References}
\begin{thebibliography}{9}

\bibitem{yenliu}
Yen Liu,
  \emph{Fourier Analysis of Numerical Algorithms for the Maxwell Equations}.
Journal of Computational Physics 124, 396-416(1996)

\bibitem{brian}
Brian Hamilton and Stefan Bilbao,
\emph{Hexagonal vs. Rectilinear Grids for Explicit Finite Difference Schemes for the Two-Dimensional Wave Equation}
Proceedings of International Congress of Acoustics(ICA), Montreal, Canada(2013)

\bibitem{kantor}
Kantorovich and Krylov,
\emph{Approximate Methods of Higher Analysis} P Noordhoff Ltd., Groningen, Netherlands.(1964)

\bibitem{southwell}
R.V Southwell, \emph{Relaxation Methods in Theoretical Physics} Oxford University Press(1946)

\end{thebibliography}
\end{frame}


\begin{frame}
\frametitle{Numerical Dispersion}

The condition for the plane waves of the form $u = e^{j\omega t} e^{j(\xi_x x + \xi_y y)}$ are solutions to the 2-D wave equation is given by the well-known dispersion relation
\begin{align}
\omega^2 = c^2 |\mathbf{\xi}|^2 = c^2 (\xi_x^2 + \xi_y^2) \label{eq:Dispersion}
\end{align}

The Phase Velocity(Wave Speed) for $|\mathbf{\xi}| > 0$ is $\omega / |\mathbf{\xi}| = c$.

The finite difference scheme approximates \eqref{eq:Dispersion} as
\begin{align}
\mathcal{D}_{tt}(\omega) = \mathcal{D}_{\Delta}(\mathbf{\xi}) \label{eq:DispersionFreqDomain}
\end{align}
for some $\mathcal{D}_{tt}: \mathbb{C} \rightarrow \mathbb{C}$ and $\mathcal{D}_{\Delta}: \mathbb{C}^2 \rightarrow \mathbb{C}$, which are Fourier symbols of the Finite-Difference Operators of the scheme. 
\begin{align}
\mathcal{D}_{tt}(\omega) &= -\frac{4}{k^2}sin^2\big(\omega \frac{k}{2}\big)\\
\mathcal{D}_{\Delta}(\mathbf{\xi}) &= -\frac{8}{3h^2} \sum_{i=1}^{3} sin^2\big((\mathbf{\xi}\cdot \mathbf{x_i})\frac{h}{2}\big)
\end{align}
\end{frame}


%----------------------------------------------------------------------------------------


\begin{frame}
\frametitle{Numerical Dispersion}
For all practical considerations, we consider Real-valued Frequencies and Wave-numbers.

To make $\mathcal{D}_{tt}$ injective, we consider the positive real wave-numbers $\xi \in \mathbb{B}$, where $\mathbb{B}$ is any Wave-Number Grid Cell.

If stability conditions are satisfied, we have
\begin{align}
\mathcal{D}_{tt}^{-1}\bigg( c^2 \mathcal{D}_{\Delta}(\mathbf{\xi})\bigg) : \mathbb{B}  \rightarrow \bigg[0, \frac{\pi}{k}\bigg]
\end{align}

Numerical Phase Velocity of as a function of wave-number:
\begin{align}
v(\mathbf{\xi}) = \frac{\omega(\mathbf{\xi})}{|\mathbf{\xi}|}, \omega(\mathbf{\xi}) = \mathcal{D}_{tt}^{-1}\bigg( c^2 \mathcal{D}_{\Delta}(\mathbf{\xi})\bigg).
\end{align}

Polar Plots: 
\begin{itemize}
\item Polar Radius: Temporal frequency $\omega$
\item Polar Angle: Angle of Propagation $\theta$ for $\mathbf{\xi} = (|\mathbf{\xi}|,\theta)$.
\end{itemize}

Numerical Phase Velocity(Wave-Speed) as a function of wave-number and temporal-frequency:
\begin{align*}
v\big(\omega(|\mathbf{\xi}|,\theta), \theta ) &= \frac{\omega(\mathbf{\xi})}{|\mathbf{\xi}|}, \text{ where }\\
\mathbf{\xi} = |\mathbf{\xi}|e^{j\theta}, \omega(\mathbf{\xi}) &= \mathcal{D}_{tt}^{-1}\bigg( c^2 \mathcal{D}_{\Delta}(\mathbf{\xi})\bigg)
\end{align*}

 
\end{frame}

\begin{frame}
\frametitle{Numerical Dispersion}

\begin{figure}
\centering
\begin{minipage}{.5\textwidth}
  \centering
\includegraphics[scale=0.2]{./images/jpgVelocity.jpg}
\label{fig:NumericalDispersion}
  \captionof{figure}{Numerical Dispersion: Wave-number tiling}
\end{minipage}%
\begin{minipage}{.5\textwidth}
  \centering
\includegraphics[scale=0.25]{./images/jpgVelocityPolar.jpg}
\label{fig:NumericalDispersionPolar}
\captionof{figure}{Numerical Phase Velocity: Polar plot}
\end{minipage}
\end{figure}



\end{frame}

%----------------------------------------------------------------------------------------
%	PRESENTATION SLIDES : END
%----------------------------------------------------------------------------------------
\end{document}